\documentclass{beamer}
\usetheme{Madrid}
\usepackage{graphicx,amsmath,algorithm2e,listings,xcolor,tabularx}
\lstset{basicstyle=\ttfamily\small,keywordstyle=\color{blue}}

\title{Understanding AI: From Concepts to Challenges}
\subtitle{A Non-Technical Deep Dive}
\author{}
\institute{}
\date{\today}

\begin{document}

\begin{frame}
\titlepage
\end{frame}

% Section 1: Basics (Slides 2-4)
\section{Core Concepts}
\begin{frame}{What is Intelligence?}
\begin{itemize}
\item \textbf{Definition}: Ability to learn, solve problems, and adapt
\item \textbf{Example}: 
\begin{itemize}
\item Humans: Learning to cook by trial/error
\item Animals: Bees finding shortest route to flowers
\end{itemize}
\end{itemize}
\end{frame}

% Slide 3: Human Intelligence
\begin{frame}{Human Intelligence}
\begin{exampleblock}{Key Traits}
\begin{itemize}
\item Creative problem-solving: Inventing wheels to move heavy objects
\item Emotional understanding: Comforting a friend
\end{itemize}
\end{exampleblock}
\end{frame}

% Slide 4: AI Definition
\begin{frame}{What is \textit{Artificial} Intelligence (AI)?}
\begin{itemize}
\item \textbf{Machines mimicking human-like thinking}
\item \textbf{Example}: 
\begin{itemize}
\item Email spam filters (like a postal worker sorting mail)
\end{itemize}
\end{itemize}
\end{frame}


% Slide 5: Types of AI
\section{Types of AI}
\begin{frame}{Rule-Based vs Data-Driven AI}
\begin{itemize}
\item \textbf{Rule-Based}: Follows fixed instructions like a cookbook
  \begin{itemize}
  \item Example: Sorting names alphabetically in a contacts list
  \end{itemize}
\item \textbf{Data-Driven}: Learns from examples like a student
  \begin{itemize}
  \item Example: Netflix suggesting shows based on your watch history
  \end{itemize}
\end{itemize}
\end{frame}

% Slide 6: Rule-Based Subtypes
\begin{frame}{Rule-Based AI Tasks}
\begin{itemize}
\item \textbf{Sorting}: Arrange data in order (e.g., price low to high)
\item \textbf{Searching}: Find items (e.g., Ctrl+F in documents)
\item \textbf{Clustering}: Group similar items (e.g., organizing books by genre)
\end{itemize}
\end{frame}

% Slide 7: Supervised Learning
\begin{frame}{Data-Driven AI: Supervised Learning}
\begin{itemize}
\item \textbf{Definition}: Learn from labeled examples (like flashcards)
\item \textbf{Example}: Predicting weather (input: humidity, wind; output: rain/sun)
\end{itemize}
\end{frame}

% Slide 8: Unsupervised Learning
\begin{frame}{Unsupervised Learning}
\begin{itemize}
\item \textbf{Definition}: Find patterns without labels (like solving a puzzle blindfolded)
\item \textbf{Example}: Grouping shoppers by buying habits (no predefined categories)
\end{itemize}
\end{frame}

% Slide 9: Semi-Supervised Learning
\begin{frame}{Semi-Supervised Learning}
\begin{itemize}
\item \textbf{Definition}: Mix of labeled + unlabeled data
\item \textbf{Example}: Diagnosing rare diseases with 10 labeled X-rays + 1,000 unlabeled
\end{itemize}
\end{frame}

% Slide 10: Self-Supervised Learning
\begin{frame}{Self-Supervised Learning}
\begin{itemize}
\item \textbf{Definition}: Create labels from the data itself
\item \textbf{Example}: Training a language model by hiding words (e.g., "The \_\_\_ barks" → "dog")
\end{itemize}
\end{frame}

% Slide 11: Reinforcement Learning
\begin{frame}{Reinforcement Learning}
\begin{itemize}
\item \textbf{Definition}: Learn by trial/error with rewards
\item \textbf{Example}: Training a robot to walk (reward for moving forward, penalty for falling)
\end{itemize}
\end{frame}

% Slide 12: Defining the Problem
\section{How AI is Built?}
\begin{frame}{How AI is Built?: Defining the Problem}
\begin{itemize}
\item \textbf{Inputs}: What the AI receives (e.g., user's age, income)
\item \textbf{Outputs}: What it produces (e.g., loan approved: yes/no)
\item \textbf{Constraints}: Limits (e.g., decision in less than 1 second)
\end{itemize}
\end{frame}

% Slide 13: Evaluation Metrics
\begin{frame}{Evaluation Metrics}
\begin{itemize}
\item \textbf{Functional}: Accuracy (e.g., 95\% correct predictions)
\item \textbf{Non-Functional}: Speed (e.g., 0.2 seconds per decision)
\end{itemize}
\end{frame}

% Slide 14: Test Cases
\begin{frame}{Test Cases}
\begin{itemize}
\item \textbf{Definition}: Scenarios to test the AI
\item \textbf{Example}: For a food-delivery app, test "order cancellation" and "payment failure"
\end{itemize}
\end{frame}

% Slide 15: Dataset Development
\begin{frame}{Dataset Development}
\begin{itemize}
\item \textbf{Size Estimation}: Guess how much data is needed
  \begin{itemize}
  \item Example: 10,000 images to train a cat/dog classifier
  \end{itemize}
\item \textbf{Raw Data Collection}: Gather initial data
  \begin{itemize}
  \item Example: Scrape restaurant reviews from Google Maps
  \end{itemize}
\end{itemize}
\end{frame}

% Slide 16: Data Filtering
\begin{frame}{Data Filtering}
\begin{itemize}
\item \textbf{Definition}: Remove useless/noisy data
\item \textbf{Example}: Deleting blurry photos from a selfie dataset
\end{itemize}
\end{frame}

% Slide 17: Data Augmentation
\begin{frame}{Data Augmentation}
\begin{itemize}
\item \textbf{Definition}: Artificially expand the dataset
\item \textbf{Example}: Flip/rotate images to create new training examples
\end{itemize}
\end{frame}

% Slide 18: Synthetic Data (Rule-Based)
\begin{frame}{Rule-Based Synthetic Data}
\begin{itemize}
\item \textbf{Definition}: Generate data using fixed rules
\item \textbf{Example}: Fake customer names (e.g., "John Smith", "Anna Lee")
\end{itemize}
\end{frame}

% Slide 19: Synthetic Data (DL-Based)
\begin{frame}{Deep Learning Synthetic Data}
\begin{itemize}
\item \textbf{Definition}: Use AI to create realistic data
\item \textbf{Example}: Generating human faces with GANs (AI art)
\end{itemize}
\end{frame}

% Slide 20: Data Preprocessing
\begin{frame}{Data Preprocessing}
\begin{itemize}
\item \textbf{Definition}: Clean/format data for AI
\item \textbf{Example}: Converting all text to lowercase for a chatbot
\end{itemize}
\end{frame}

% Slide 21: Train/Val/Test Split
\begin{frame}{Splitting Data}
\begin{itemize}
\item \textbf{Train}: Practice data (70\%)
\item \textbf{Validation}: Tune AI (15\%)
\item \textbf{Test}: Final exam (15\%)
\end{itemize}
\end{frame}

% Slide 22: Model Selection
\begin{frame}{Model Selection}
\begin{itemize}
\item \textbf{Rule-Based}: Fixed logic (e.g., flowchart for medical diagnosis)
\item \textbf{Data-Driven}: Learned patterns (e.g., neural network for stock prediction)
\end{itemize}
\end{frame}

% Slide 23: Hyperparameter Tuning
\begin{frame}{Hyperparameter Tuning}
\begin{itemize}
\item \textbf{Definition}: Adjusting "knobs" for better performance
\item \textbf{Example}: Setting how fast the AI learns (learning rate)
\end{itemize}
\end{frame}

% Slide 24: Model Training
\begin{frame}{Model Training}
\begin{itemize}
\item \textbf{Process}: Teach AI using training data
\item \textbf{Tools}: Track experiments with MLflow/Weights \& Biases
\end{itemize}
\end{frame}

% Slide 25: Model Validation
\begin{frame}{Model Validation}
\begin{itemize}
\item \textbf{Definition}: Check performance before deployment
\item \textbf{Example}: Testing a self-driving car in a simulator
\end{itemize}
\end{frame}

% Slide 26: Deployment - Hardware
\begin{frame}{Deployment: Hardware}
\begin{itemize}
\item \textbf{Edge}: Run on devices (e.g., face unlock on phones)
\item \textbf{Cloud}: Run on servers (e.g., weather prediction on AWS)
\end{itemize}
\end{frame}

% Slide 27: Deployment - Software
\begin{frame}{Deployment: Software}
\begin{itemize}
\item \textbf{Webservice}: API endpoint (e.g., skin cancer detection website)
\item \textbf{Containers}: Package AI in Docker for easy sharing
\end{itemize}
\end{frame}

% Slide 28: Hardware Optimizations
\begin{frame}{Hardware Optimizations}
\begin{itemize}
\item \textbf{TPUs}: Google’s custom chips for faster training
\item \textbf{Tensor Cores}: Specialized units in GPUs for math
\end{itemize}
\end{frame}

% Slide 29: Software Optimizations
\begin{frame}{Software Optimizations}
\begin{itemize}
\item \textbf{Quantization}: Reduce number precision (e.g., 32-bit → 8-bit)
\item \textbf{Kernel Fusion}: Combine operations for speed
\end{itemize}
\end{frame}

% Slide 30: Parallelism
\begin{frame}{Parallelism: Training and inference}
\begin{itemize}
\item \textbf{Data}: Split data across GPUs
\item \textbf{Pipeline}: Process different stages simultaneously
\end{itemize}
\end{frame}

% Slide 31: Caching
\begin{frame}{Caching}
\begin{itemize}
\item \textbf{KV Caching}: Reuse past computations (e.g., ChatGPT speedup)
\end{itemize}
\end{frame}

% Slide 32: Model Monitoring
\begin{frame}{Model Monitoring}
\begin{itemize}
\item \textbf{Drift Detection}: Check if data changes over time
\item \textbf{Example}: Spam filter failing due to new slang
\end{itemize}
\end{frame}

% Slide 33: Model Retraining
\begin{frame}{Model Retraining}
\begin{itemize}
\item \textbf{Active Learning}: Prioritize uncertain data
\item \textbf{Example}: Labeling ambiguous medical scans first
\end{itemize}
\end{frame}

% Slide 34: Trusted AI - Security
\section{Retrospect on AI}
\begin{frame}{Challenges in AI: Security}
\begin{itemize}
\item \textbf{Watermarking}: Embed hidden markers in AI outputs
\item \textbf{Adversarial Defense}: Block malicious inputs
\end{itemize}
\end{frame}

% Slide 43: Security - Model Encryption
\begin{frame}{Encryption}
\begin{itemize}
\item \textbf{Definition}: Protect AI models from theft (like a locked safe)
\item \textbf{Example}: 
  \begin{itemize}
  \item Encrypting a self-driving car’s AI model to prevent hacking
  \end{itemize}
\end{itemize}
\end{frame}


% Slide 35: Deepfakes
\begin{frame}{Deepfakes}
\begin{itemize}
\item \textbf{Definition}: AI-generated fake media
\item \textbf{Example}: Fake video of a politician making false claims
\end{itemize}
\end{frame}

% Slide 36: Explainability
\begin{frame}{Explainability}
\begin{itemize}
\item \textbf{Definition}: Understand why AI made a decision
\item \textbf{Example}: Highlighting words that caused spam classification
\end{itemize}
\end{frame}

% Slide 37: Formal Verification
\begin{frame}{Formal Verification}
\begin{itemize}
\item \textbf{Definition}: Mathematically prove AI safety
\item \textbf{Example}: Ensuring a robot arm never moves too fast
\end{itemize}
\end{frame}

% Slide 38: Model Alignment
\begin{frame}{Model Alignment}
\begin{itemize}
\item \textbf{Definition}: Match AI goals with human values
\item \textbf{Example}: Chatbot refusing harmful requests
\end{itemize}
\end{frame}

% Continue adding slides 39-43 as needed



% Slide 39: Transformer on Chip
\begin{frame}{Performant AI: Transformer on Chip}
\begin{itemize}
\item \textbf{Definition}: Custom chips designed for transformer models (common in language models)
\item \textbf{Example}: 
  \begin{itemize}
  \item Google’s TPU v4 optimized for BERT/GPT-style models
  \item Etched ASICs
  \end{itemize}
\end{itemize}

\end{frame}

% Slide 40: Kernel Graphs
\begin{frame}{CUDA Kernel Graphs}
\begin{itemize}
\item \textbf{Definition}: Predefined execution plans for speed (like a recipe)
\item \textbf{Example}: 
  \begin{itemize}
  \item TensorFlow using graphs to run models faster
  \end{itemize}
\end{itemize}
\end{frame}

% Slide 41: Tensor Parallelism
\begin{frame}{Tensor Parallelism}
\begin{itemize}
\item \textbf{Definition}: Split model layers across devices (like teamwork)
\item \textbf{Example}: 
  \begin{itemize}
  \item Training GPT-3 across 100+ GPUs simultaneously
  \end{itemize}
\end{itemize}
\end{frame}

% Slide 42: Caching - Browser
\begin{frame}{Caching}
\begin{itemize}
\item \textbf{Definition}: Store data locally to avoid repeated downloads
\item \textbf{Example}: 
  \begin{itemize}
  \item Voice assistant saving common responses offline
  \item Design of KV caches in transformer models to save GPU memory and still runtime benefits
  \end{itemize}
\end{itemize}
\end{frame}



\begin{frame}{Development Challenges}
\begin{itemize}
\item \textbf{Data Quality}: 
\begin{itemize}
\item Example: Training self-driving cars with biased urban data fails in rural areas
\end{itemize}
\item \textbf{Computational Costs}: 
\begin{itemize}
\item Example: Training ChatGPT-4 required \$100M+ in computing power
\end{itemize}
\end{itemize}
\end{frame}

\begin{frame}{Deployment Challenges}
\begin{itemize}
\item \textbf{Hardware Limitations}: 
\begin{itemize}
\item Example: Smart speakers struggle with complex queries due to small memory
\end{itemize}
\item \textbf{Ethical Concerns}: 
\begin{itemize}
\item Example: Facial recognition misidentifying people with darker skin tones
\end{itemize}
\end{itemize}
\end{frame}

\begin{frame}{Monitoring Challenges}
\begin{itemize}
\item \textbf{Data Drift}: 
\begin{itemize}
\item Example: COVID-19 pandemic made old health prediction models inaccurate
\end{itemize}
\item \textbf{Adversarial Attacks}: 
%% https://www.overleaf.com/project/68287e55aca4d5a1136834fa
\begin{itemize}
\item Example: Adding invisible noise to stop signs to confuse self-driving cars
\end{itemize}
\end{itemize}
\end{frame}

\begin{frame}{Adoption Challenges}
\begin{itemize}
\item \textbf{User Trust}: 
\begin{itemize}
\item Example: Patients rejecting AI cancer diagnoses despite 98\% accuracy
\end{itemize}
\item \textbf{Regulatory Compliance}: 
\begin{itemize}
\item Example: GDPR laws forcing companies to explain AI decisions
\end{itemize}
\end{itemize}
\end{frame}

\begin{frame}{Environmental Impact}
\begin{itemize}
\item \textbf{Energy Consumption}: 
\begin{itemize}
\item Example: Training one AI model emits 5x car's lifetime CO\textsubscript{2}
\end{itemize}
\item \textbf{E-Waste}: 
\begin{itemize}
\item Example: Obsolete AI chips filling landfills
\end{itemize}
\end{itemize}
\end{frame}

\begin{frame}{Future Challenges}
\begin{itemize}
\item \textbf{Job Displacement}: 
\begin{itemize}
\item Example: AI writing tools reducing demand for junior copywriters
\end{itemize}
\item \textbf{Superintelligence Risks}: 
\begin{itemize}
\item Example: Hypothetical AI pursuing goals conflicting with human values
\end{itemize}
\end{itemize}
\end{frame}

% Conclusion
\begin{frame}{Key Takeaways}
\begin{itemize}
\item AI mimics human intelligence but faces real-world limitations
\item Development requires balancing technical and ethical factors
\item Continuous monitoring ensures reliability
\end{itemize}
\end{frame}

\end{document}
